\documentclass{article}
\usepackage[utf8]{inputenc}
\usepackage[headings]{fullpage}
\usepackage[utf8]{inputenc}
\usepackage{graphicx}
\usepackage{amsmath}
\usepackage{siunitx}
\usepackage{amssymb}
\usepackage{mathtools}
\usepackage[margin=1in]{geometry}


\title{Chemistry 172}
\author{$\text{I} \Sigma \Pi$}
\date{Winter 2016}

\begin{document}

\maketitle

\section{General Equations}


\subsection{Entropy Equations}

\begin{table}[h]
\centering
\caption{Equations for Entropy}
\label{my-label}
\def\arraystretch{1.5}
\begin{tabular}{|l|l|l|l|l|}
\hline
Vary Temperature & Vary Pressure & Vary Volume   \\
\hline
$\Delta S = C_\text{v/p}\ln{\frac{T_f}{T_i}}$   & $\Delta S = nR\ln{\frac{P_i}{P_f}}$ & $\Delta S = nR\ln{\frac{V_f}{V_i}}$ \\
\hline
\end{tabular}
\end{table}

\begin{equation*}
S = k\ln{W}
\end{equation*}

\begin{equation*}
S_{\text{surr}} = - q_{\text{reaction}}/T = - \frac{\Delta H_{sys}}{T}
\end{equation*}

\[
T = \frac{\Delta U}{\Delta S} = (\frac{dU}{dS})_v
\]

\subsubsection{Entropy for energy states} 
Choose lower levels first, and go up correspondingly. Multiply the options. Note that you don't have to do the last step because once you chose everything but the last stage, then there is no choice there. 



\subsection{Enthalpy Equations}
\begin{align*}
\begin{split}
\
\Delta H &= \Delta U + P \Delta V \\ 
\
\Delta H &= \Delta U + \Delta n RT\\
\
\Delta H &= \frac{C_p}{\Delta T}\\
\
\Delta U &= \frac{C_V}{\Delta T}\\
\
C_p &= C_v + nR\\
\
w_\text{sys} & = - \int_{V_f}^{V_i} P dV\\
\
dG & = w_{non-exp}\\
\end{split}
\end{align*}

Do note for IE and EA, they are both defined as the energy required/released to remove/add an electron respectively to a neutral species.
\[
X + e^- \rightarrow X^- + \text{energy} (EA)
\]

\[
X + \text{energy} \rightarrow X^+ + e^- (IE)
\]



\begin{table}[h]
\centering
\caption{Equations for Work}
\label{my-label}
\def\arraystretch{1.5}
\begin{tabular}{|l|l|l|l|l|}
\hline
Constant Pressure & Constant Temperature & Constant Volume   \\
\hline
$w_\text{sys} = - P_{\text{ext}} \Delta V$   & $w_\text{sys} = - n R T\ln{\frac{V_\text{f}}{V_\text{i}}}$ & 0 \\
\hline
\end{tabular}
\end{table}

\begin{table}[h!]
\centering
\caption{Ideal Molecules}
\label{my-label}
\def\arraystretch{1.5}
\begin{tabular}{|l|l|l|l|l|l|}
\hline
Molecule & Translation & Rotation & $C_\text{v}$ & $C_\text{p}$ & Internal Energy  \\
\hline
Atom & 3 & 0 & $\frac{3}{2}R$ & $\frac{5}{2}R$ & $\frac{3}{2}\ nRT$\\
\hline
Linear & 3 & 2 & $\frac{5}{2}R$& $\frac{7}{2}R$ & $\frac{5}{2}\ nRT$\\
\hline
Non-Linear & 3 & 3 & 3R & 4R & 3\ nRT\\
\hline
\end{tabular}
\end{table}

\section{Carnot/Heat Engines}

No engine working between two given heat reserviors can be more efficient than a reversible engine working between these two reserviors. The cycle is known as the Carnot cycle. No engine can have a greater efficiency than a Carnot engine because all steps are reversible.

\begin{equation*}
\epsilon = 1 - \frac{Q_\text{c}}{Q_\text{h}}
\end{equation*}

\begin{equation*}
Q_\text{h} = W_\text{by gas},\ Q_\text{c} = W_\text{on gas} 
\end{equation*}

\begin{equation*}
Q_\text{h/c} = nRT_\text{h/c} \ln\frac{V_\text{2}}{V_\text{1}}
\end{equation*}

\begin{equation*}
\epsilon_{\text{C}} = 1 - \frac{T_\text{c}}{T_\text{h}} \ \text{(Carnot Efficiency)}
\end{equation*}

\begin{figure}[h]
\centering
\includegraphics[scale=0.5]{carnot.png}
\caption{Carnot Cycle}
\end{figure}

\section{Equilibrium}
\begin{equation*}
\Delta S_{\text{total}} = 0
\end{equation*}

\begin{equation*}
\Delta G = \Delta H - T \Delta S
\end{equation*}

At equilibirum, $\Delta G =0$ and this yields:

\[
\Delta S = \Delta H/T 
\]
\begin{table}[h!]
\centering
\caption{Gibbs Free Energy}
\label{my-label}
\def\arraystretch{1.5}
\begin{tabular}{|l|l|l|l|l|l|}
\hline
$\Delta H$ & $\Delta S$ & $\Delta G$ & Spontaneous?  \\
\hline
$<$0 & $>$0 & $<$0 & yes \\
\hline
$>$0 & $<$0 & $>$0 & no\\
\hline
$<$0 & $<$0 & ? & yes if $|T\Delta S| < \Delta$ H  \\
\hline 
$>$0 & $>$0 & ? & yes if $\Delta H < T \Delta$ S \\
\hline
\end{tabular}
\end{table}

\section{Clausius Inequality}
\[
\Delta U = q + w
\]

$\Delta U$ is same for paths that start and end at the same point. $w_{rev} > w_{irr}$

\[
\Delta U = \text{constant}, q_{irr} > q_{rev}
\]

\section{Vapor Pressure}
\[
T = \frac{\Delta H_{vap}}{\Delta S_{vap}}
\]

Pressure at which the liquid and vapor are in dynamic equilibrium is called the vapor pressure. 
\[
P_{ext}=P_{vapor}
\]



Once the external pressure meets the vapor pressure at that temperature, the liquid will start to boil. Hence, the vapor pressure is dependent on the temperature, and the boiling point is dependent on external pressure. Note that these two relationships are not the same. Vapor pressure is also dependent on intermolecular forces.

We determine vapor pressure using proportional reasoning:
\[
P_{vap} = \frac{\Delta}{760mm} \times atm
\]


\[
\ln{\frac{P'_{vap}}{P^{\circ}_{vap}}} = \frac{\Delta H^{\circ}_{vap}}{R} (\frac{1}{T_B}-\frac{1}{T'})
\]

\begin{center}
Where X' refers to the new conditions and $X^{\circ}$ refers to the known values at the substance's boiling point. For example, for water, we can set P = 1 bar and $T_B = 373K$, and we can determine its vapor pressure at any other temperature we care to determine it at.
\end{center}

\[
\Delta H^{\circ}_{vap} - T \Delta S^{\circ}_{vap} = RT \ln \frac{P^{\circ}}{P_{vap}}
\]

Where $P^{\circ}$ is 1 bar with $P_{vap}$ being the vapor pressure at temperature T.\\

\[
\Delta G_r = \Delta G_r^{\circ}+RT\ln Q
\]

Note that $\Delta G_r$ goes to zero at equilibrium and Q becomes K. Some uses are to determine:

\[
\Delta G_{vap} (P) = \Delta G_{vap}^{\circ}+RT\ln \frac{P}{P_b}
\]

When $Q>K$, reactants are formed. When $Q<K$, products are formed.\\

Clausius-Clapeyron Equation:
\[
P_{vap} = Ae^{-\Delta H^{\circ}_{vap}/RT},\ A=e^{\Delta H^{\circ}_{vap}/RT_B}
\]
Assumptions: 
\begin{enumerate}
\item$\Delta H_{vap}$ and $\Delta S_{vap}$ remain approximately constant.
\item Gas is ideal
\item The molar pressure of the water is much greater than the molar pressure of the gas. 
\end{enumerate}

\begin{figure}[hbt!]
\centering
\includegraphics[scale=0.5]{phase.png}
\caption{Phase Diagram}
\end{figure}

\section{Chemical Equilibrium}

\[
a A + b B \leftrightarrow c C + d D
\]


The most general case for the equilibrium constant would be the fraction of the concentration of the products raised to the power of their coefficients over the concentration of the reactants. The following ratios are also known by the law of mass ratios.

\[
K = \frac{\text{products}}{\text{reactants}}
\]

\subsubsection{Le Chatelier's Principle}
When stress is applied to a system in dynamic equilibrium, the equilibrium tends to adjust to minimise the effect of the stress.

\begin{enumerate}
\item Change in Reactants
\item Change in Pressure
\item Change in Temperature
\end{enumerate}

\subsection{Pressure}

\[
K_p= \frac{P_C^c P_D^d}{P_A^a P_B^b}
\]

Where P is the partial pressure of the gas.

\subsection{Concentration}

\[
K_C = \frac{[C]^c [D]^d}{[A]^a [B]^b}
\]

Note that solids do not enter the equation here.

\subsection{Additonal Equations}
\[
\Delta G = \Delta G^{\circ} + RT \ln Q
\]

Where Q is the activity at any time. K can be substituted for Q at equilibrium. At equilibrium, $\Delta G$ is zero.

\[
\Delta H - T\Delta S = -RT \ln K
\]

\[
K = e^{-\frac{\Delta G^{\circ}}{RT}},\ \text{(At equilibrium)}
\]


Changing to an unknown equilibrium constant, where K', T' are the new conditions:
\[
\ln \frac{K'}{K} = \frac{\Delta H}{R} (\frac{1}{T} - \frac{1}{T'})
\]

This is known as the Van't Hoff equation. Note the dependence of K on T. It is not dependant on concentration, pressure, volume, or anything else other than temperature.

\section{Acids and Bases}

\[
\text{pH} = \text{pKa} + \log[\text{A}^-]/[\text{HA}]
\]

Use ICE table to determine changes of concentration. Note that you can use overall change in moles and use the final equilibrium conditions to determine the final moles. From there, $x$ can be obtained and the changes can be calculated.\\

pH has nothing to do with how acidic a substance is, it merely refers to the concentration in the solution. Acidity is determined by how strongly the acid dissociates. One way to describe $K_a$ is 'concentration of protons produced per unit concentration of acid'. It is independent of the amount of acid used whereas pH could be dependent on the amount of acid used.\\

The strongest buffer point is where pH = pka.\\

Note that when a strong base or acid is added. React the acid/base first before calculating the equilibrium point! Remember that the acid or base completely dissociates, so remember to remove the appropriate amount of $H^+$ or $OH^-$ (and their salts!) before applying the ICE table.\\

\section{Redox Reactions}
Reductions always result in:
\begin{enumerate}
\item Gain in electrons
\item Loss of oxygen
\item Lowering of oxidation state
\end{enumerate}
Oxidations always result in:
\begin{enumerate}
\item Loss of electrons
\item Gain of oxygen
\item Increase in oxidation state

\end{enumerate}

\[
\Delta G^{\circ} = - n_r F \Delta V^{\circ}
\]

where $n_r$ refers to the number of electrons tranferred in each reaction

\subsection{Cell Diagrams}

Oxidation is on the left, and reduction is on the right. Solid is always on the outside. Different states are separated by a line. Reactants in the same phase are separated by a comma. Note that it is always written in a way that it flows from left to right

\[
Cu \rightarrow Cu^{2+} + 2e^- \text{(oxidation)}
\]
\[
Ag^+ + e^- \rightarrow Ag \text{(reduction)}
\]

\begin{center}
$Cu|Cu^{2+}||Ag^{+}|Ag$
\end{center}

\[
\Delta V^{\circ} = \Delta V^{\circ}_{right} - \Delta V^{\circ}_{left}
\]
Written as the standard reduction potential.
\\
\\
In order to standarise measurements, we introduce the standard hydrogen electrode and use it as the anode, and set the $V^{\circ}$ to be zero.
\[
Pt|H_2|H^+||Cu^{+}|Cu
\]

We use a Pt electrode if there are only gases or liquids in the half-cell. In this case, $V^{\circ}_{right} = V^{\circ}_{cell}$. If they are in the same phase, we use a comma to separate them.

\[
\Delta G^{\circ} = - nF \Delta V^{\circ}
\]

Where n is the number of electrons transferred in the balanced redox reaction. 

\subsubsection{Deprotonation}

\[
\% deprotonation = \frac{[H^+]}{[HA]}
\]

\subsubsection{Nernst Equation}
\[
\Delta V = \Delta V^{\circ} - \frac{RT}{nF} \ln Q
\]

where Q:

\[
Q = \frac{[products]}{[reactants]}
\]

\section{Rates}
\subsection{0th Order}
\[
R = k
\]
\[
[A] = kt
\]
\subsection{1st Order}
\[
R = k[A]
\]
\[
[A] = [A]_0 e^{-kt}
\]

Half lives (only for first order):

\[
\frac{\ln 2}{k} = t_{1/2}
\]

\subsection{2nd Order}
\[
R = k [A]^2
\]
\[
\frac{1}{[A]} = \frac{1}{[A]_0} +kt
\]


\section{Credits}
\begin{enumerate}
\item Carnot Cycle, Physics for Science and Engineers
\item Phase Diagram, Chemical Principles
\end{enumerate}

\section{Contributors}
\begin{enumerate}
\item Alex-Ortiz
\item Ari Jacobson

\end{enumerate}
















\end{document}
