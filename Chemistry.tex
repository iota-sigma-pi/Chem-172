\documentclass{article}
\usepackage[utf8]{inputenc}
\usepackage[headings]{fullpage}
\usepackage[utf8]{inputenc}
\usepackage{graphicx}
\usepackage{amsmath}
\usepackage{siunitx}
\usepackage{amssymb}
\usepackage{mathtools}
\usepackage[margin=1in]{geometry}


\title{Chemistry 172}
\author{$\text{I} \Sigma \Pi$}
\date{Winter 2016}

\begin{document}

\maketitle

\section{General Equations}

\subsection{Entropy Equations}

\begin{table}[h]
\centering
\caption{Equations for Entropy}
\label{my-label}
\def\arraystretch{1.5}
\begin{tabular}{|l|l|l|l|l|}
\hline
Vary Temperature & Vary Pressure & Vary Volume   \\
\hline
$\Delta S = C_\text{v/p}\ln{\frac{T_f}{T_i}}$   & $\Delta S = nR\ln{\frac{P_i}{P_i}}$ & $\Delta S = nR\ln{\frac{V_f}{V_i}}$ \\
\hline
\end{tabular}
\end{table}

\begin{equation*}
S = k\ln{W}
\end{equation*}


\begin{equation*}
S_{\text{surr}} = - q_{\text{reaction}}/T 
\end{equation*}



\subsection{Enthalpy Equations}
\begin{align*}
\begin{split}
\
\Delta H &= \Delta U + P \Delta V \\ 
\
\Delta H &= \Delta U + \Delta n RT\\
\
\Delta H &= \frac{C_p}{\Delta T}\\
\
\Delta U &= \frac{C_V}{\Delta T}\\
\
C_p &= C_v + nR\\
\
w_\text{sys} & = - \int_{V_f}^{V_i} P dV
\end{split}
\end{align*}

\begin{table}[h]
\centering
\caption{Equations for Work}
\label{my-label}
\def\arraystretch{1.5}
\begin{tabular}{|l|l|l|l|l|}
\hline
Constant Pressure & Constant Temperature & Constant Volume   \\
\hline
$w_\text{sys} = - P_{\text{ext}} \Delta V$   & $w_\text{sys} = - n R T\ln{\frac{V_\text{f}}{V_\text{i}}}$ & 0 \\
\hline
\end{tabular}
\end{table}

\begin{table}[h]
\centering
\caption{Ideal Molecules}
\label{my-label}
\def\arraystretch{1.5}
\begin{tabular}{|l|l|l|l|l|l|}
\hline
Molecule & Translation & Rotation & $C_\text{v}$ & $C_\text{p}$ & Internal Energy  \\
\hline
Atom & 3 & 0 & $\frac{3}{2}R$ & $\frac{5}{2}R$ & $\frac{3}{2}\ nRT$\\
\hline
Linear & 3 & 2 & $\frac{5}{2}R$& $\frac{7}{2}R$ & $\frac{5}{2}\ nRT$\\
\hline
Non-Linear & 3 & 3 & 3R & 4R & 3\ nRT\\
\hline
\end{tabular}
\end{table}

\section{Carnot/Heat Engines}

No engine working between two given heat reserviors can be more efficient than a reversible engine working between these two reserviors. The cycle is known as the Carnot cycle. No engine can have the a greater efficiency than a Carnot engine because all steps are reversible.

\begin{equation*}
\epsilon = 1 - \frac{Q_\text{c}}{Q_\text{h}}
\end{equation*}

\begin{equation*}
Q_\text{h} = W_\text{by gas},\ Q_\text{c} = W_\text{on gas} 
\end{equation*}

\begin{equation*}
Q_\text{h/c} = nRT_\text{h/c} \frac{V_\text{2}}{V_\text{1}}
\end{equation*}

\begin{equation*}
\epsilon_{\text{C}} = 1 - \frac{T_\text{c}}{T_\text{h}} \ \text{(Carnot Efficiency)}
\end{equation*}

\section{Equilibrium}
\begin{equation*}
\Delta S_{\text{total}} = 0
\end{equation*}

\begin{equation*}
\Delta G = \Delta H - T \Delta S
\end{equation*}

\end{document}
