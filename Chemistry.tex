\documentclass{article}
\usepackage[utf8]{inputenc}
\usepackage[headings]{fullpage}
\usepackage[utf8]{inputenc}
\usepackage{graphicx}
\usepackage{amsmath}
\usepackage{siunitx}
\usepackage{amssymb}
\usepackage{mathtools}
\usepackage[margin=1in]{geometry}


\title{Chemistry 172}
\author{$\text{I} \Sigma \Pi$}
\date{Winter 2016}

\begin{document}

\maketitle

\section{General Equations}


\subsection{Entropy Equations}

\begin{table}[h]
\centering
\caption{Equations for Entropy}
\label{my-label}
\def\arraystretch{1.5}
\begin{tabular}{|l|l|l|l|l|}
\hline
Vary Temperature & Vary Pressure & Vary Volume   \\
\hline
$\Delta S = C_\text{v/p}\ln{\frac{T_f}{T_i}}$   & $\Delta S = nR\ln{\frac{P_i}{P_i}}$ & $\Delta S = nR\ln{\frac{V_f}{V_i}}$ \\
\hline
\end{tabular}
\end{table}

\begin{equation*}
S = k\ln{W}
\end{equation*}


\begin{equation*}
S_{\text{surr}} = - q_{\text{reaction}}/T 
\end{equation*}



\subsection{Enthalpy Equations}
\begin{align*}
\begin{split}
\
\Delta H &= \Delta U + P \Delta V \\ 
\
\Delta H &= \Delta U + \Delta n RT\\
\
\Delta H &= \frac{C_p}{\Delta T}\\
\
\Delta U &= \frac{C_V}{\Delta T}\\
\
C_p &= C_v + nR\\
\
w_\text{sys} & = - \int_{V_f}^{V_i} P dV
\end{split}
\end{align*}

\begin{table}[h]
\centering
\caption{Equations for Work}
\label{my-label}
\def\arraystretch{1.5}
\begin{tabular}{|l|l|l|l|l|}
\hline
Constant Pressure & Constant Temperature & Constant Volume   \\
\hline
$w_\text{sys} = - P_{\text{ext}} \Delta V$   & $w_\text{sys} = - n R T\ln{\frac{V_\text{f}}{V_\text{i}}}$ & 0 \\
\hline
\end{tabular}
\end{table}

\begin{table}[h!]
\centering
\caption{Ideal Molecules}
\label{my-label}
\def\arraystretch{1.5}
\begin{tabular}{|l|l|l|l|l|l|}
\hline
Molecule & Translation & Rotation & $C_\text{v}$ & $C_\text{p}$ & Internal Energy  \\
\hline
Atom & 3 & 0 & $\frac{3}{2}R$ & $\frac{5}{2}R$ & $\frac{3}{2}\ nRT$\\
\hline
Linear & 3 & 2 & $\frac{5}{2}R$& $\frac{7}{2}R$ & $\frac{5}{2}\ nRT$\\
\hline
Non-Linear & 3 & 3 & 3R & 4R & 3\ nRT\\
\hline
\end{tabular}
\end{table}

\section{Carnot/Heat Engines}

No engine working between two given heat reserviors can be more efficient than a reversible engine working between these two reserviors. The cycle is known as the Carnot cycle. No engine can have the a greater efficiency than a Carnot engine because all steps are reversible.

\begin{equation*}
\epsilon = 1 - \frac{Q_\text{c}}{Q_\text{h}}
\end{equation*}

\begin{equation*}
Q_\text{h} = W_\text{by gas},\ Q_\text{c} = W_\text{on gas} 
\end{equation*}

\begin{equation*}
Q_\text{h/c} = nRT_\text{h/c} \frac{V_\text{2}}{V_\text{1}}
\end{equation*}

\begin{equation*}
\epsilon_{\text{C}} = 1 - \frac{T_\text{c}}{T_\text{h}} \ \text{(Carnot Efficiency)}
\end{equation*}

\begin{figure}[h]
\centering
\includegraphics[scale=0.5]{carnot.png}
\caption{Carnot Cycle}
\end{figure}

\section{Equilibrium}
\begin{equation*}
\Delta S_{\text{total}} = 0
\end{equation*}

\begin{equation*}
\Delta G = \Delta H - T \Delta S
\end{equation*}

At equilibirum, $\Delta G =0$ and this yields:

\[
\Delta S = \Delta H/T 
\]
\begin{table}[h!]
\centering
\caption{Gibbs Free Energy}
\label{my-label}
\def\arraystretch{1.5}
\begin{tabular}{|l|l|l|l|l|l|}
\hline
$\Delta H$ & $\Delta S$ & $\Delta G$ & Spontaneous?  \\
\hline
$<$0 & $>$0 & $<$0 & yes \\
\hline
$>$0 & $<$0 & $>$0 & no\\
\hline
$<$0 & $<$0 & ? & yes if $|T\Delta S| < \Delta$ H  \\
\hline 
$>$0 & $>$0 & ? & yes if $\Delta H < T \Delta$ S \\
\hline
\end{tabular}
\end{table}

\section{Clausius Inequality}
\[
\Delta U = q + w
\]

$\Delta U$ is same for paths that start and end point. $w_{rev} > w_{irr}$

\[
\Delta U = \text{constant}, q_{irr} > q_{rev}
\]

\section{Vapor Pressure}

Pressure at which the liquid and vapor are in dynamic equilibrium is called the vapor pressure. Once the external pressure meets the vapor pressure at that temperature, the liquid will start to boil. Hence, the vapor pressure is dependent on the temperature, and the boiling point is dependent on external pressure. Note that these two relationships are not the same. \\
\\
Vapor pressure is also dependent on intermolecular forces.\\

\[
\ln{\frac{P'_{vap}}{P^{\circ}_{vap}}} = \frac{\Delta H^{\circ}_{vap}}{R} (\frac{1}{T_B}-\frac{1}{T'})
\]
\begin{center}
Where X' refers to the new conditions and $X^{\circ}$ refers to the known values at the substance's boiling point.
\end{center}

Clausius-Clapeyron Equation:
\[
P_{vap} = Ae^{-\Delta H^{\circ}_{vap}/RT},\ A=e^{\Delta H^{\circ}_{vap}/RT_B}
\]
Assumptions: 
\begin{enumerate}
\item$\Delta H_{vap}$ and $\Delta S_{vap}$ remain approximately constant.
\item Gas is ideal
\item The molar pressure of the water is much greater than the molar pressure of the gas. 
\end{enumerate}

\begin{figure}[hbt!]
\centering
\includegraphics[scale=0.5]{phase.png}
\caption{Phase Diagram}
\end{figure}

\section{Chemical Equilibrium}

\[
a A + b B \leftrightarrow c C + d D
\]


The most general case for the equilibrium constant would be the fraction of the concentration of the products raised to the power of their coefficients over the concentration of the reactants. The following ratios are also known by the law of mass ratios.

\[
K = \frac{\text{products}}{\text{reactants}}
\]

\subsection{Pressure}

\[
K_p= \frac{P_C^c P_D^d}{P_A^a P_B^b}
\]

Where P is the partial pressure of the gas.

\subsection{Concentration}

\[
K_C = \frac{[C]^c [D]^d}{[A]^a [B]^b}
\]

Note that solids do not enter the equation here.

\subsection{Additonal Equations}
\[
\Delta G = \Delta G^{\circ} + RT \ln Q
\]

Where Q is the activity at any time. K can be substituted for Q at equilibrium. At equilibrium, $\Delta G$ is zero.

\[
K = e^{-\frac{\Delta G^{\circ}}{RT}},\ \text{(At equilibrium)}
\]


Changing to an unknown equilibrium constant, where K', T' are the new conditions:
\[
\ln \frac{K'}{K} = \frac{\Delta H}{R} (\frac{1}{T} - \frac{1}{T'})
\]

Note the dependence of K on T. It is not dependant on concentration, pressure, volume, or anything else other than temperature.

\section{Acids and Bases}

\[
\text{pH} = \text{pKa} + \log[\text{A}^-]/[\text{HA}]
\]

Use ICE table to determine changes of concentration. Note that you can use overall change in moles and use the final equilibrium conditions to determine the final moles. From there, $x$ can be obtained and the changes can be calculated.

pH has nothing to do with how acidic a substance is, it merely refers to the concentration in the solution. Acidity is determined by how strongly the acid dissociates. One way to describe $K_a$ is 'concentration of protons produced per unit concentration of acid'. It is independent of the amount of acid used whereas pH could be dependent on the amount of acid used.

\section{Credits}
\begin{enumerate}
\item Carnot Cycle, Physics for Science and Engineers
\item Phase Diagram, Chemical Principles
\end{enumerate}

\section{Contributors}
\begin{enumerate}
\item Alex-Ortiz

\end{enumerate}
















\end{document}
